\documentclass[12pt, preprint]{aastex}
\usepackage{graphicx}	% For figures
\usepackage{natbib}	% For citep and citep
\usepackage{amsmath}	% for \iint
\usepackage{bbm}
\usepackage[breaklinks]{hyperref}	% for blackboard bold numbers
\usepackage{hyperref}
\hypersetup{colorlinks}
\usepackage{color}
\usepackage{morefloats}
\definecolor{darkred}{rgb}{0.5,0,0}
\definecolor{darkgreen}{rgb}{0,0.5,0}
\definecolor{darkblue}{rgb}{0,0,0.5}
\hypersetup{ colorlinks,
linkcolor=darkblue,
filecolor=darkgreen,
urlcolor=darkred,
citecolor=darkblue }

\DeclareMathOperator*{\argmax}{arg\,max}

\newcommand{\beq}{\begin{equation}}
\newcommand{\eeq}{\end{equation}}

\begin{document}

\title{Likelihood free inference with measurements of galaxy clustering, group richness, and number denisty }

\begin{abstract}

The Halo Occupation Distribution model (HOD) is a powerful prescription for 
understanding how galaxies populate dark matter halos.
In this work, we use the Approximate Bayesian Computation method, 
ABC, in order to explore the parameters of the halo model with mock
measurements of galaxy number denisty, galaxy clustering, 
and group richness. These observables are forward-modeled in the ABC
by populating halo catalogs in the Bolshoi N-body dark matter simulation 
with the HOD model and making the above measurements on the generated
galaxy catalogs. We use the publicly available software, 
HALOTOOLS to perform the forward-modeling and make the mock measurements.
We find that including group richness in our set of observables leads to stronger
constraints on the HOD parameters that govern the population of satellite galaxies.
Our ABC analysis also successfully constrains HOD parameters for these same
measurements made in SDSS data and our constrained values are consistent with
previous attempts in the literature.

The Halo Occupation Distribution (HOD) model is a powerful prescription of how galaxies populate 
dark matter halos that has had much success in reproducing large scale galaxy clustering and understanding
galaxy evolution within the context of the halo model. Constraints on the parameters of the HOD model have 
so far been imposed using the standard practice of Bayesian inference with likelihood estimation that 
assumes a functional (often times Gaussian) form of the likelihood. Likelihood free inference methods such 
as the Approximate Bayesian Computation (ABC) makes Bayesian inference possible without ever having to calculate the likelihood. 

In this work, we present ABC in the context of constraining the parameters of the HOD model. For a mock 
galaxy catalog with known HOD parameters, we use ABC on measurements of a set of observables generated
through HALOTOOLs, a publicly available ..., to infer tight constraints that reproduce the known HOD paramters. 
Since ABC does not require computing the likelihood, we include observables beyond galaxy clustering such as 
the galaxy number density and the galaxy group richness, for which accurately constructing a likelihood is intractable. 
We demonstrate that adding these measurements to our set of observables leads to stronger constraints on the HOD parameters, 
specifically the ones that govern the satellite galaxy population. Moreover, the constraints we infer that 
reproduce the known HOD parameters illustrate the success of ABC in the context of constraining HOD parameters. 

We further extend our ABC method with the above set of observables to constrain HOD parameters for the Sloan 
Digital Sky Survey Data Release 7 observed galaxy catalog. The constraints we find on the HOD parameters are consistent with 
(or stronger than) previous constraints in the literature. (Ultimately include more interesting results).

\end{abstract}

\section{Introduction}

Parameter estimation in cosomlogy is commonly done with Bayesian inference. 
In the Bayesian framework, constraints on cosmological parameters are achieved by 
sampling from the posterior distribution, the probability of the parameters of the 
model given the data. Posteriors are evaluated by multiplying the prior probability 
distribution of the paramters of the model with the likelihood function, the probability 
of the observed data given the parameters of the model.
Therefore, Bayesian inference relies on the evaluation of the likelihood function.
 
Evaluation of the likelihood requires on making assumption about the functional 
form of the likelihood. It is often assumed that the likelihood is Gaussian distribution 
with a given covariance matrix that needs to be evaluated from simulating a large set of 
mock catalogs, or other techniques such as bootstrap. In order to 
compute the likelihood funciton, one needs to invert a high dimensional covariance 
matrix which makes the likelihood evaluation and therefore sampling from the posterior 
computationally intractable.

Approximate Bayesian Computation, ABC, is a method that eliminates the need for likelihood. 
In recent years, this technique has been used in a number of cosmological studies, such as 
the inference of cosmological parameters from modelling of the distances of supernovae type Ia, 
and modelling of weak lensing peak maps. Exploring the parameter space with ABC algorithm 
requires three essential ingredients: (1) a prior pdf on the parameters that we wish to estimate, 
(2) a forward model (simulator) of the observations, and (3) a distance metric for comparing 
the model and the observed data. In problems where likelihood evaluation is not analytically or 
computationally tractable, this algorithm bypasses the need for likelihood evaluation.

To date, different variations of ABC algorithm have been introduced  and implemented 
in the literature. One of the earliest methods is likelihood free Markov Chain Monte 
Carlo. In this algorithm, an initial draw from the prior is kept, if the distance between 
the simulated model corresponding to that parameter and the observed data is less than a threshold. 
In the consequent steps, the proposed parameters are accepted if \emph{first} they satisfy the 
distance requirement, and \emph{second} the Metropolis-Hasting (MH) acceptance ratio evaluated for the 
proposal is less than one. In the MH acceptance ratio, the likelihood is replaced with some transition 
kernel. 

Based on importance sampling, several modifications to this algorithm have been proposed. 
These methods are called ABC Sequential Monte Carlo (ABC-SMC), or ABC Population Monte Carlo 
(ABC-PMC). These algorithms rely on sampling a large set of parameters from the prior pdf, and assigning 
equal weights to them as an initial step. In the subsequent steps, sampling is modified by 
proposing intermediate distribution built from the previous sample of parameters and their 
corresponding weights. Parameters are updated by sampling from the intermediate distribution 
and are accepted if they pass the distance requirements. Furthermore, weights are updated according 
to the intermediate distribution. 

Talk about importance sampling here ...
Sampling from the parameter space in ABC algorithm can become more efficient by 
employing adaptive importance sampling... 

\section{Method}

\subsection{Approximate Bayesian Computation}

\subsubsection{basics}

Talk about generic abc ...

\subsubsection{ABC with adaptive importance sampling}






\section{Data}




\section{Results}


\begin{thebibliography}{70}


\end{thebibliography}


\end{document}
